\documentclass[12pt, bachelor, sepnum, expert]{ur-thesis}
\usepackage{blindtext}
\bibliography{literatur.bib}
\thmnum{chapter}
\begin{document}\setthtitle{Titel der Arbeit}
%	\addzweitgutachter{asdklfj}
	\setaddress{Erzgebirgstr. 7\\93164 Laaber}
	\setabgdatum{22.04.3000}
	\setbearbzeit{29.05.2000}{14.03.2001}
	\setgutachter{friedl}
	\defaulttitlepage

	
	
	\begin{zusammenfassung}
	Test
	Dann gibt es auch verschiedene Abs\"atze und manchmal macht der ganze Text, wie in diesem Beispiel, keinen Sinn. Wir betrachten noch die Abbildungen
	und 
	sowie die Tabelle \thtitle
	
	
	
	\end{zusammenfassung}
		\tableofcontents
	\begin{lemma}
		Aussage
	\end{lemma}

\begin{satz}
	Aussage
	
	%\iftitle
	\ifx\thtitle\relax
		Titel
	\else
	 	kein Titel
	\fi
\end{satz}

\begin{bemerkung}
	Aussage \parencite{dehnen2009}
\end{bemerkung}

\begin{definition}
	Definition
\end{definition}


%\begin{vartheorem}{öalskdjf}
%	äldkjfalskdjfaösldkfj
%\end{vartheorem}

\begin{proof}
	Beweis
\end{proof}

\begin{theorem}
	Theorem
\end{theorem}

\begin{vartheorem}{plain}{VariableTheorem}
	Inhalt...
\end{vartheorem}


\begin{vartheorem}{plain}{VariableTheorem}
	Inhalt...
\end{vartheorem}


\blinddocument
\end{document}