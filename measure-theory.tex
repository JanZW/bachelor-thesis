\chapter{Prerequisites}
\section{Review of Measure Theory and Topology}
%set M^1(X) of all probability measures on the Borel-\sigma-Algebra of X
%M^1 with weak*-topology is compact

In this section we will remind the reader briefly of some definitions and results from measure theory and topology that will be of use later.

Throughout, we will use the following notation.
\begin{notation}Let $\Omega$ be a set.
	\begin{enumerate}
		\item Let $A\subset \Omega$ be a subset. Then $A^c$ denotes the complement of $A$ in $\Omega$.
		\item $\mathcal{P}(\Omega)$ shall denote the set of subsets of $\Omega$. 
	\end{enumerate}
\end{notation}
\noindent Recall the following definition as can be found in \cite{friedl:ana-3}.
%\begin{definition}
%	Let $\Omega$ be a set. We say $\mathcal{A}\subset \mathcal{P}(\Omega)$ is a \emph{set-algebra} on $\Omega$, if the following statements hold:
%	\begin{enumerate}
%		\item $\emptyset\in\mathcal{A}$,
%		\item $A\in \mathcal{A}\Rightarrow A^c\in \mathcal{A}$,
%		\item $A,B\in \mathcal{A}\Rightarrow A\cup B\in \mathcal{A}$
%	\end{enumerate}
%\end{definition}

\begin{definition}
	Let $\Omega$ be a set. We say $\mathcal{A}$ is a \emph{$\sigma$-Algebra on $\Omega$}, if the following statements are true:
	\begin{enumerate}
		\item $\emptyset\in \mathcal{A}$.
		\item $A\in \mathcal{A}\Rightarrow A^c\in \mathcal{A}$.
		\item The union of countably many sets in $\mathcal{A}$ is in $\mathcal{A}$.
	\end{enumerate}
\end{definition}

\begin{definition}
	For $S\subset\mathcal{P}(\Omega)$, we call 
	\[
	\langle S\rangle^\sigma:=\text{Intersection of all $\sigma$-Algebras on $\Omega$ that contain $S$}
	\]
	the $\sigma$-Algebra \emph{generated by} $S$.
\end{definition}

\begin{definition}
	Let $(X,\mathcal{T})$ be a topological space. We call the $\sigma$-Algebra generated by all open sets in $X$, i.e. $\mathcal{B}=\langle \mathcal{T} \rangle^\sigma$, the \emph{Borel-$\sigma$-Algebra}.
\end{definition}


\begin{definition}Let $\Omega$ be a set and $\mathcal{A}$ a $\sigma$-Algebra on $\Omega$. 
\begin{enumerate}%definition regular measure
%\item We call $(\Omega,\mathcal{A})$ a measure space.%todo no! measure space?
\item A function 
	\[
	\mu:\mathbb{A}\to \R\cup \{\pm\infty\}
	\]
	satisfying
	\begin{enumerate}
		\item $\mu(\emptyset)=0$
		\item For any pairwise disjoint family $(A_k)_{k\in\N}$  of sets in $\mathcal{A}$
		\[
		\mu\left(\bigcup_{k=1}^\infty A_k\right)=\sum_{k=1}^{\infty}\mu(A_k)
		\]	
	\end{enumerate}\enlargethispage{1cm}
is called a \emph{signed measure on $(\Omega,\mathcal{A})$}.

\item We say $\mu$ is a \emph{(positive) measure}, if it is a signed measure with the property 
\[
\mu(X)\geq 0\quad \forall X\in \mathcal{A}.
\]

\item Let $X$ be a Hausdorff space, $\mathcal{A}$ a $\sigma$-algebra with $\mathcal{B}\subset\mathcal{A}$.  We say $\mu$ is a \emph{(positive) regular measure} if it is a measure such that for all $A\in \mathcal{A}$
\[
\mu(A)=\inf\{\mu(U)\colon A\subset U, U \text{ open}\}=\sup\{\mu(K)\colon K\subset A, K\text{ compact} \}.
\] 
and we say $\mu$ is a \emph{signed regular measure} if 
\[
|\mu|(A):=\sup_{\mathfrak{D}}\sum_{E\in \mathfrak{D}}|\mu(E)|,
\]
is a positive regular measure, where the supremum is taken over all decompositions of $A$ into finitely many disjoint sets.

\item We say $\mu$ is a \emph{probability measure}, if it is a positive regular measure and
\[
\mu(\Omega)=1.
\]

\item We say $\mu$ is \emph{finite} if it does not take on the values $\pm\infty$ for any set.
\end{enumerate}
\end{definition}

\begin{notation}
	Let $X$ be a topological space and $\mathcal{B}(X)$ the Borel-$\sigma$-Algebra on $X$. We write
	\begin{enumerate}
	\item	$\M(X)=\{\mu\mid \mu$ is a finite, regular signed Borel measure on $X\}$.
	\item $\M^+(X)=\{\mu\mid \mu$ is a finite, regular Borel measure on $X\}$.
	\item $\M^1(X)=\{\mu\mid\mu$ is a Borel probability measure on $X \}$.
	\end{enumerate}
\end{notation}

We will equip $\M^1(X)$ with a suitable topology later (namely the weak*-topology).

Note that
\[
\M^1(X)\subset \M^+(X)\subset \M(X)
\]
In particular the following clearly holds.

\begin{lemma}
	Equipped with the usual operations of addition and scalar multiplication, the set $\M(X)$ is a real vector space, and the spaces $\M^+(X)$ and $\M^1(X)$ are convex subsets of $\M(X)$.
\end{lemma}

We now define a special kind of measures, namely those whose support is a singleton. 
\begin{definition}
	%A measure with finite support is called an \emph{atomic measure}. If the support consists of only one point, then such a (atomic) measure $\delta_x$ is called a \emph{Dirac measure}, in particular
	The measures assigning weight one to a single point are called \emph{Dirac measures}, i.e.
	\[
	\delta_x(A)=\begin{cases}
	1&\text{if }x\in A\\
	0&\text{if }x\not\in A
	\end{cases},
	\] 
	are called \emph{Dirac measures}.
\end{definition}
We will see in \autoref{thm:finite-support} that Dirac measures are important for approximating more general measures.

We conclude this review of measure theory with a reminder of the following theorem as can be found in \cite[Theorem 4.11, 4.13]{friedl:ana-3} or \cite[S.18]{cohn}
\begin{theorem}[Theorem of Carathéodory]\label{thm:measure-extension}
	Let $X$ be a set, let $\mu^\ast$ be an outer measure on $X$, and let $\mathcal{A}_{\mu^\ast}$ be the collection of all $\mu^\ast$-measurable subsets of $X$. Then
	\begin{enumerate}
		\item $\mathcal{A}_{\mu^\ast}$ is a $\sigma$-slgebra, and
		\item the restriction of $\mu^\ast$ to $\mathcal{A}_{\mu^\ast}$ is a measure on $\mathcal{A}_{\mu^\ast}$.
	\end{enumerate}	
\end{theorem}

%The following results are well known and therefore no proof will be given here. We refer the interested reader to \cite{friedl:ana3}.
%
%\begin{lemma}
%	There does not exist a measure on $\mathcal{P}(\R^n)$ for $n\in \N$.
%\end{lemma}

%\begin{definition}
%	A function $f: X\to \R$ is called measurable if the sets $f^{-1}(-\infty,a]$ are measurable for all $a\in \R$.
%\end{definition}

%\begin{lemma}
%	All real valued continuous functions are measurable.
%\end{lemma}

%\section{Topology and measure theory}
%topological groups, haar measure ... nah
It is well known that any topology induces a concept of what it means for a sequence to converge with respect to that topology. This convergence can be defined as follows.
 

\begin{definition}
	Let $(X,\mathcal{T})$ be a topological space. A sequence $(a_i)_{i\in\N}$ in $X$ \emph{converges to $a\in X$ with respect to the topology $\mathcal{T}$} if for every open neighborhood $U$ of $a$ there exists $N\in \N$ such that for all $n>N$ we have $a_n\in U$.
\end{definition}

Also, the following connection between closed sets and sequences is well known.

\begin{lemma}
	Let $X$ be a topological space and $A\subset X$ a closed set. Let $(a_i)_{i\in \N}$ be a sequence in $A$, converging to $a$. Then $a\in A$.
\end{lemma}

This gives rise to the question if the converse might be true in the sense that defining what it means for a sequence to converge already uniquely determines the topology on the space. As it turns out, this is generally not the case for non-metrizable spaces.
For example, let $X$ be a set with uncountably many points. Then the discrete and the co-countable topology do not coincide, however a sequence converges with respect to the discrete topology if and only if it converges in the co-countable topology.

For a similar statement to hold we need the concept of nets, which are in some sense generalized sequences.

\begin{definition}[\cite{choquet:analysis}, p. 48-50]\mbox{}
	\begin{enumerate}
		\item A \emph{directed set} $D$ is a partially ordered set such that for each $m,n\in D$ there is a $p\in D$ so that $p\geq m$ and $p\geq n$.
		\item Let $E$ be a set and $D$ a directed set. A \emph{net on $E$} is a mapping $f:D\to E$.
		\item A net $x_a$ \emph{converges  to $a\in E$} if and only if for every neighborhood $U$ of $a$, there is an $a_0$ such that $a\geq a_0$ implies $x_a\in U$.
		\item We say a net \emph{converges} if it converges to some point in $E$.
	\end{enumerate}
\end{definition}


\begin{proposition}
	Let $E$ be a topological space. Then $A\subset E$ is closed if and only if for any net $x_a$ with $x_a\in A$ which converges to $a\in E$, we have $a\in A$.
\end{proposition}
\begin{proof}
	If $A$ is closed then $a\in A$, as $a$ is a cluster point of the set $\{x_a\mid a\in D\}$. Conversely, if $A$ were not closed there would be a cluster point $a\in E\setminus A$. For $U$ any neighborhood of $a$, choose $x_U\in U\cap A$. Then $x_U$ forms a net in $A$ converging to $a$.
\end{proof}

We can now define the weak*-topology on $\M(E)$ using nets (\cite[see][]{choquet:analysis}).
\begin{definition}
	We define the weak*-topology on $\M(E)$ to be the unique topology such that for a net $(\mu_i)$ in $\M(E)$ converges to a $\mu\in \M(E)$ if and only if 
	\[
	\int_E\phi(x)\mu_i(\dx)\rightarrow\int_E\phi(x)\mu(\dx)\quad \forall \phi\in \mathcal{K}(E,\R),
	\]
	where $\mathcal{K}(E,\R)$ is the set of all continuous functions from $E$ to $\R$ with compact support.
\end{definition}

%\begin{definition}%moved
%	Let $(X,\chi)$ be a measurable space and let $\mu:\chi\to \R\cup\{-\infty,+\infty\}$ be a function. If $\mu(\emptyset)=0$ and 
%	\[
%	\mu\left(\bigcup_{n=1}^\infty A_n\right)=\sum_{n=1}^{\infty}\mu(A_n)
%	\]
%	for all disjoint sequences $(A_n)\in \chi$ then $\mu$ is a \emph{signed measure}. In addition, if $\mu(A)\geq 0$ for all $A\in \chi$ then $\mu(a)\geq 0$ for all $A\in \chi$ then $\mu$ is a (positive) \emph{measure}. Further, if $\mu$ is a measure such that $\mu(X)=1$ then $\mu(X)=1$ then $\mu$ is a \emph{probability measure}. A (signed) measure is called a \emph{finite (signed) measure} if it does neither take the value $+\infty$ or $-\infty$.
%\end{definition}

%\begin{definition}%moved
%	Let $X$ be a topological space. Then the Borel-$\sigma$-algebra is the $\sigma$-algebra generated by the open subsets of $X$
%\end{definition}

%\begin{notation}%moved to measure-theory.tex
%	We will use the following notation throughout this thesis:
%	\begin{align*}
%		\M(X)&:=\{\text{finite signed Borel measures on }X\}\\
%		\M^+(X)&:=\{\text{finite Borel measures on }X\}\\
%		\M^1(X)&:=\{\text{Borel probability measures on }X\}
%	\end{align*}
%\end{notation}


%nets
%weak*-topology
%\begin{definition}%Wikipedia schwach-*-topologie
%	Jedes Element $x$ aus einem normierten oder allgemeiner lokalkonvexen $\mathbb{K}$-Vektorraum $E$ ($\mathbb{K}$ ist hier $\R$ oder $\C$) definiert durch die Formel $\hat{x}(f):=f(x)$ ein lineares Funktional auf dem topologischen Dualraum $E^\prime$. Die schwach-*-Topologie ist definiert als die schächste Topologie auf $E^\prime$, die all diese Abbildungen $\hat{x}: E^\prime\to \mathbb{K}$ stetig macht.
%	
%	Eine etwas konkretere Definition erhält man durch die Angabe einer Umgebungsbasis. Für $f\in E^\prime$ bilden die Mengen
%	
%	\[U_{f}(x_{1},\ldots ,x_{n},\epsilon ):=\{g\in E';|f(x_{j})-g(x_{j})|<\epsilon ,j=1,\ldots ,n\},\]
%	
%	wobei $x_{1},\ldots ,x_{n}\in E,n\in {{\mathbb N}},\epsilon >0,$ 
%	eine Umgebungsbasis schwach-*-offener Mengen von f. Die 
%	schwach-*-Topologie wird oft mit 
%	w* bezeichnet, nach der englischen Bezeichnung weak-*-topology, oder mit $\sigma (E\,',E)$, um die Herkunft als Initialtopologie anzudeuten
%\end{definition}
%
%\begin{definition}%initial topology choquet
%	Let $E$ be a set, $\{E_i\}$ a family of topological spaces and $f_i:E\to E_i$ mappings. The initial (or projective) topology on $E$ is the coarsest in which each $f_i$ is continuous.
%\end{definition}
%
%This topology is generated by $\{f_i^{-1}(U)\mid U\subset E_i \text{ open}\}$.
%
%\begin{definition}%weak* topology choquet
%	Let $E$ be a locally compact Hausdorff space and $\M(E)$ the collection of Radon measures on $E$. For each $\phi\in \mathcal{K}(E,\R)$ consider the map $f_\phi: \M(E)\to \R$ defined by $f(\mu)=\mu(\phi)$. The initial topology on $\M(E)$ with respect to the maps $\{f_\phi\mid \phi\in \mathcal{K}(E,\R)\}$ is called the vague topology (sometimes called the weak*-topology).
%	
%	This topology is characterized by: for any net $\mu_i, (\mu_i\to \mu)$ if and only if (for all $\phi \in \mathcal{K}(E;\R)$), where $\mathcal{K}(E,\R)$ is the set of continuous functions form $E$ to $\R$ with compact support.
%\end{definition}

%\begin{theorem}[\cite{choquet:analysis}, theorem 12.6]\label{thm:compactness}
%	Let $E$ be a locally compact Hausdorff space, and $X\subset \mathcal{M}(E)$. Then the following are equivalent:
%	\begin{enumerate}
%		\item $\mathrm{cl}(X)$ is compact (in the vague topology)
%		\item $X$ is vaguely bounded
%		\item for each $K\subset E$ compact, there is an $N_K$ such that for all $f$ with support in $K$
%		\[
%		|\mu(f)|\leq N_K\|f\|
%		\]
%		for all $\mu in X$ (that is, $X$ is strongly bounded.)
%		\item for each $K\subset E$ compact there is a $P_K$ such that
%		\[
%		|\mu|(K)\leq P_K
%		\]
%		for all $\mu\in K$.
%	\end{enumerate}
%\end{theorem}%TODO proof
%
%
%
%\begin{theorem}[Theorem of approximation for measures,\cite{choquet:analysis}]\label{thm:density}
%	Let $E$ be a locally compact Hausdorff space. Then the vector space generated by $\{\epsilon_x\}$ for all $x\in E$ is dense in $\M(E)$,
%\end{theorem}


%uniform topology



