%\section{Topology and measure theory}
%topological groups, haar measure ... nah
It is well known that any topology induces a concept of what it means for a sequence to converge with respect to that topology. This convergence can be defined as follows.
 

\begin{definition}
	Let $(X,\mathcal{T})$ be a topological space. A sequence $(a_i)_{i\in\N}$ in $X$ \emph{converges to $a\in X$ with respect to the topology $\mathcal{T}$} if for every open neighborhood $U$ of $a$ there exists $N\in \N$ such that for all $n>N$ we have $a_n\in U$.
\end{definition}

Also, the following connection between closed sets and sequences is well known.

\begin{lemma}
	Let $X$ be a topological space and $A\subset X$ a closed set. Let $(a_i)_{i\in \N}$ be a sequence in $A$, converging to $a$. Then $a\in A$.
\end{lemma}

This gives rise to the question if the converse might be true in the sense that defining what it means for a sequence to converge already uniquely determines the topology on the space. As it turns out, this is generally not the case for non-metrizable spaces.
For example, let $X$ be a set with uncountably many points. Then the discrete and the co-countable topology do not coincide, however a sequence converges with respect to the discrete topology if and only if it converges in the co-countable topology.

For a similar statement to hold we need the concept of nets, which are in some sense generalized sequences.

\begin{definition}[\cite{choquet:analysis}, p. 48-50]\mbox{}
	\begin{enumerate}
		\item A \emph{directed set} $D$ is a partially ordered set such that for each $m,n\in D$ there is a $p\in D$ so that $p\geq m$ and $p\geq n$.
		\item Let $E$ be a set and $D$ a directed set. A \emph{net on $E$} is a mapping $f:D\to E$.
		\item A net $x_a$ \emph{converges  to $a\in E$} if and only if for every neighborhood $U$ of $a$, there is an $a_0$ such that $a\geq a_0$ implies $x_a\in U$.
		\item We say a net \emph{converges} if it converges to some point in $E$.
	\end{enumerate}
\end{definition}


\begin{proposition}
	Let $E$ be a topological space. Then $A\subset E$ is closed if and only if for any net $x_a$ with $x_a\in A$ which converges to $a\in E$, we have $a\in A$.
\end{proposition}
\begin{proof}
	If $A$ is closed then $a\in A$, as $a$ is a cluster point of the set $\{x_a\mid a\in D\}$. Conversely, if $A$ were not closed there would be a cluster point $a\in E\setminus A$. For $U$ any neighborhood of $a$, choose $x_U\in U\cap A$. Then $x_U$ forms a net in $A$ converging to $a$.
\end{proof}

We can now define the weak*-topology on $\M(E)$ using nets (\cite[see][]{choquet:analysis}).
\begin{definition}
	We define the weak*-topology on $\M(E)$ to be the unique topology such that for a net $(\mu_i)$ in $\M(E)$ converges to a $\mu\in \M(E)$ if and only if 
	\[
	\int_E\phi(x)\mu_i(\dx)\rightarrow\int_E\phi(x)\mu(\dx)\quad \forall \phi\in \mathcal{K}(E,\R),
	\]
	where $\mathcal{K}(E,\R)$ is the set of all continuous functions from $E$ to $\R$ with compact support.
\end{definition}

%\begin{definition}%moved
%	Let $(X,\chi)$ be a measurable space and let $\mu:\chi\to \R\cup\{-\infty,+\infty\}$ be a function. If $\mu(\emptyset)=0$ and 
%	\[
%	\mu\left(\bigcup_{n=1}^\infty A_n\right)=\sum_{n=1}^{\infty}\mu(A_n)
%	\]
%	for all disjoint sequences $(A_n)\in \chi$ then $\mu$ is a \emph{signed measure}. In addition, if $\mu(A)\geq 0$ for all $A\in \chi$ then $\mu(a)\geq 0$ for all $A\in \chi$ then $\mu$ is a (positive) \emph{measure}. Further, if $\mu$ is a measure such that $\mu(X)=1$ then $\mu(X)=1$ then $\mu$ is a \emph{probability measure}. A (signed) measure is called a \emph{finite (signed) measure} if it does neither take the value $+\infty$ or $-\infty$.
%\end{definition}

%\begin{definition}%moved
%	Let $X$ be a topological space. Then the Borel-$\sigma$-algebra is the $\sigma$-algebra generated by the open subsets of $X$
%\end{definition}

%\begin{notation}%moved to measure-theory.tex
%	We will use the following notation throughout this thesis:
%	\begin{align*}
%		\M(X)&:=\{\text{finite signed Borel measures on }X\}\\
%		\M^+(X)&:=\{\text{finite Borel measures on }X\}\\
%		\M^1(X)&:=\{\text{Borel probability measures on }X\}
%	\end{align*}
%\end{notation}


%nets
%weak*-topology
%\begin{definition}%Wikipedia schwach-*-topologie
%	Jedes Element $x$ aus einem normierten oder allgemeiner lokalkonvexen $\mathbb{K}$-Vektorraum $E$ ($\mathbb{K}$ ist hier $\R$ oder $\C$) definiert durch die Formel $\hat{x}(f):=f(x)$ ein lineares Funktional auf dem topologischen Dualraum $E^\prime$. Die schwach-*-Topologie ist definiert als die schächste Topologie auf $E^\prime$, die all diese Abbildungen $\hat{x}: E^\prime\to \mathbb{K}$ stetig macht.
%	
%	Eine etwas konkretere Definition erhält man durch die Angabe einer Umgebungsbasis. Für $f\in E^\prime$ bilden die Mengen
%	
%	\[U_{f}(x_{1},\ldots ,x_{n},\epsilon ):=\{g\in E';|f(x_{j})-g(x_{j})|<\epsilon ,j=1,\ldots ,n\},\]
%	
%	wobei $x_{1},\ldots ,x_{n}\in E,n\in {{\mathbb N}},\epsilon >0,$ 
%	eine Umgebungsbasis schwach-*-offener Mengen von f. Die 
%	schwach-*-Topologie wird oft mit 
%	w* bezeichnet, nach der englischen Bezeichnung weak-*-topology, oder mit $\sigma (E\,',E)$, um die Herkunft als Initialtopologie anzudeuten
%\end{definition}
%
%\begin{definition}%initial topology choquet
%	Let $E$ be a set, $\{E_i\}$ a family of topological spaces and $f_i:E\to E_i$ mappings. The initial (or projective) topology on $E$ is the coarsest in which each $f_i$ is continuous.
%\end{definition}
%
%This topology is generated by $\{f_i^{-1}(U)\mid U\subset E_i \text{ open}\}$.
%
%\begin{definition}%weak* topology choquet
%	Let $E$ be a locally compact Hausdorff space and $\M(E)$ the collection of Radon measures on $E$. For each $\phi\in \mathcal{K}(E,\R)$ consider the map $f_\phi: \M(E)\to \R$ defined by $f(\mu)=\mu(\phi)$. The initial topology on $\M(E)$ with respect to the maps $\{f_\phi\mid \phi\in \mathcal{K}(E,\R)\}$ is called the vague topology (sometimes called the weak*-topology).
%	
%	This topology is characterized by: for any net $\mu_i, (\mu_i\to \mu)$ if and only if (for all $\phi \in \mathcal{K}(E;\R)$), where $\mathcal{K}(E,\R)$ is the set of continuous functions form $E$ to $\R$ with compact support.
%\end{definition}

%\begin{theorem}[\cite{choquet:analysis}, theorem 12.6]\label{thm:compactness}
%	Let $E$ be a locally compact Hausdorff space, and $X\subset \mathcal{M}(E)$. Then the following are equivalent:
%	\begin{enumerate}
%		\item $\mathrm{cl}(X)$ is compact (in the vague topology)
%		\item $X$ is vaguely bounded
%		\item for each $K\subset E$ compact, there is an $N_K$ such that for all $f$ with support in $K$
%		\[
%		|\mu(f)|\leq N_K\|f\|
%		\]
%		for all $\mu in X$ (that is, $X$ is strongly bounded.)
%		\item for each $K\subset E$ compact there is a $P_K$ such that
%		\[
%		|\mu|(K)\leq P_K
%		\]
%		for all $\mu\in K$.
%	\end{enumerate}
%\end{theorem}%TODO proof
%
%
%
%\begin{theorem}[Theorem of approximation for measures,\cite{choquet:analysis}]\label{thm:density}
%	Let $E$ be a locally compact Hausdorff space. Then the vector space generated by $\{\epsilon_x\}$ for all $x\in E$ is dense in $\M(E)$,
%\end{theorem}


%uniform topology


