	
\begin{zusammenfassung}
	1964 veröffentlichte Oliver Alfred Gross seine Abhandlung \glqq The Rendezvous Value of Metric Spaces\grqq \cite{gross}. Er beschrieb, wie durch Anwendung wohlbekannter Aussagen der Spieltheorie interessante Ergebnisse in anderen Teilbereichen der Mathematik bewiesen werden können. Er führt dies anhand der \glqq Rendezvous Values\grqq{} metrischer Räume vor.
	Die Hauptaussage ist die folgende:
	\begin{quotation}
		Für jeden kompakten, zusammenhängenden nicht-leeren metrischen Raum $X$ existiert eine eindeutige Konstante $K$ mit der Eigenschaft, dass es zu jeder endlichen Familie $A:=(x_i)_{i\in I}$ von Punkten in $X$ einen weiteren Punkt $p$ in $X$ gibt, so dass das arithmetische Mittel der Distanzen der Punkte in $A$ zu $p$ gleich $K$ ist.
	\end{quotation}
	Wir werden sehen, dass sich diese Aussage auf eine allgemeinere Klasse von Räumen verallgemeinern lässt.
	
	Ziel dieser Arbeit ist es einen Zugang zu diesen und verwandten Aussagen zu schaffen, der nur wenige Kenntnisse voraussetzt, die über die üblichen Aussagen der Analysis III heraus gehen. Es wird lediglich die Kenntnis des Satzes Hahn-Banach benötigt. Ein grundlegendes Verständnis für Banach Räume ist hilfreich für Kapitel \ref{chap:further}, aber nicht zwingend erforderlich. 
	In Kapitel \ref{computations} werden einige Beispiele vorgestellt und Aussagen zur Berechnung bewiesen.
\end{zusammenfassung}
