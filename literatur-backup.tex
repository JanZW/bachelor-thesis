Automatically generated by Mendeley Desktop 1.19.4
Any changes to this file will be lost if it is regenerated by Mendeley.

BibTeX export options can be customized via Options -> BibTeX in Mendeley Desktop

@book{choquet:analysis,
author = {Choquet, Gustave.},
isbn = {9780805369618},
publisher = {Benjamin},
title = {{Lectures on analysis / 1. Integration and topological vector spaces.}},
year = {1969}
}
@article{strantzen,
author = {Strantzen, John},
doi = {10.1017/S0004972700005815},
file = {::},
journal = {Bull. Aust. Math. Soc.},
month = {dec},
number = {3},
pages = {321--330},
publisher = {Cambridge University Press},
title = {{An average distance result in Euclidean $n$ -space}},
url = {https://www.cambridge.org/core/product/identifier/S0004972700005815/type/journal_article},
urldate={2019-09-11},
volume = {26},
year = {1982}
}
@techreport{friedl:ana-3,
author = {Friedl, Stefan},
title = {{Analysis III-Wintersemester 2015}},
urldate={2019-09-11},
url = {https://www.uni-regensburg.de/Fakultaeten/nat_Fak_I/friedl/papers/2015-2016_analysis-III-regensburg.pdf}
}

@article{kokkendorff,
author = {Kokkendorff, Simon L.},
doi = {10.1016/j.difgeo.2008.04.012},
file = {::},
journal = {Differ. Geom. its Appl.},
month = {dec},
number = {6},
pages = {638--644},
title = {{Characterizing the round sphere by mean distance}},
url = {https://linkinghub.elsevier.com/retrieve/pii/S0926224508000363},
urldate={2019-09-11},
volume = {26},
year = {2008}
}
@article{Morris1983,
author = {Morris, Sidney A. and Nickolas, Peter},
file = {::},
journal = {Arch. Math},
pages = {459--463},
title = {{On the average distance property of compact connected metric spaces}},
url = {http://www.sidneymorris.net/Morris67.pdf},
urldate={2019-09-11},
volume = {40},
year = {1983}
}
@article{cleary:numbers-of-shapes,
author = {Cleary, Joan and Morris, Sidney A. and Yost, David},
doi = {10.2307/2323675},
file = {:C\:/Users/Jan/Documents/Studium/Rendezvous number/Cleary-Morris-Yost260-275.pdf:pdf},
issn = {00029890},
journal = {Am. Math. Mon.},
month = {apr},
number = {4},
pages = {260--275},
title = {{Numerical Geometry-Numbers for Shapes}},
url = {https://www.jstor.org/stable/2323675?origin=crossref},
urldate={2019-09-11},
volume = {93},
year = {1986}
}
@article{wolf:spheres,
author = {Wolf, Reinhard},
doi = {10.1007/BF01201787},
issn = {0003-889X},
journal = {Arch. der Math.},
month = {apr},
number = {4},
pages = {338--344},
publisher = {Birkh{\"{a}}user-Verlag},
title = {{On the average distance property of spheres in Banach spaces}},
url = {http://link.springer.com/10.1007/BF01201787},
urldate={2019-09-11},
volume = {62},
year = {1994}
}
@article{nickolas-yost:euclidean,
author = {Nickolas, Peter and Yost, David},
doi = {10.1007/BF01190235},
file = {:C\:/Users/Jan/Documents/Studium/Rendezvous number/Nickolas-Yost1988_Article_TheAverageDistancePropertyForS.pdf:pdf},
issn = {0003-889X},
journal = {Arch. der Math.},
month = {apr},
number = {4},
pages = {380--384},
title = {{The average distance property for subsets of euclidean space}},
url = {http://link.springer.com/10.1007/BF01190235},
urldate={2019-09-11},
volume = {50},
year = {1988}
}
@article{wolf:certain-banach,
author = {Wolf, Reinhard},
doi = {10.1017/S0004972700030616},
journal = {Bull. Aust. Math. Soc.},
month = {feb},
number = {1},
pages = {147--160},
publisher = {Cambridge University Press},
title = {{Averaging distances in certain Banach spaces}},
url = {https://www.cambridge.org/core/product/identifier/S0004972700030616/type/journal_article},
urldate={2019-09-11},
volume = {55},
year = {1997}
}
@article{hinrichs:banach2,
abstract = {A Banach space X has the average distance property (ADP) if there exists a unique real number r such that for each positive integer n and all x_1,...,x_n in the unit sphere of X there is some x in the unit sphere of X such that 1/n \sum_{k=1}^n ||x_k-x|| = r. We show that l_p does not have the average distance property if p>2. This completes the study of the ADP for l_p spaces.},
author = {Hinrichs, Aicke and Wenzel, J.},
file = {::},
journal = {Bull. Aust. Math. Soc.},
month = {feb},
title = {{The average distance property of classical Banach spaces II}},
url = {http://arxiv.org/abs/math/0202093},
urldate={2019-09-11},
volume = {65},
year = {2002}
}
@article{blumenthal,
author = {Blumenthal, L. M. and Wahlin, G. E.},
doi = {10.1090/S0002-9904-1941-07565-8},
file = {::},
issn = {0002-9904},
journal = {Bull. Am. Math. Soc.},
month = {oct},
number = {10},
pages = {771--778},
title = {{On the spherical surface of smallest radius enclosing a bounded subset of $n$-dimensional euclidean space}},
url = {http://www.ams.org/journal-getitem?pii=S0002-9904-1941-07565-8},
urldate={2019-09-11},
volume = {47},
year = {1941}
}
@article{nickolas-wolf:quasihypermetric3,
abstract = {Let $(X, d)$ be a compact metric space and let $\mathcal{M}(X)$ denote the space of all finite signed Borel measures on $X$. Define $I \colon \mathcal{M}(X) \to \R$ by \[ I(\mu) = \int_X \int_X d(x,y) d\mu(x) d\mu(y), \] and set $M(X) = \sup I(\mu)$, where $\mu$ ranges over the collection of signed measures in $\mathcal{M}(X)$ of total mass 1. This paper, with two earlier papers [Peter Nickolas and Reinhard Wolf, Distance geometry in quasihypermetric spaces. I and II], investigates the geometric constant $M(X)$ and its relationship to the metric properties of $X$ and the functional-analytic properties of a certain subspace of $\mathcal{M}(X)$ when equipped with a natural semi-inner product. Specifically, this paper explores links between the properties of $M(X)$ and metric embeddings of $X$, and the properties of $M(X)$ when $X$ is a finite metric space.},
author = {Nickolas, Peter and Wolf, Reinhard},
file = {::},
journal = {Math. Nachrichten},
month = {sep},
pages = {747--760},
title = {{Distance Geometry in Quasihypermetric Spaces. III}},
url={https://onlinelibrary.wiley.com/doi/abs/10.1002/mana.200810216},
urldate={2019-09-11},
volume = {284},
year = {2008}
}
@article{hinrichs:johns-ellipsoid,
author = {Hinrichs, Aicke},
file = {:C\:/Users/Jan/AppData/Local/Mendeley Ltd./Mendeley Desktop/Downloaded/Hinrichs - 2001 - The average distance property of classical banach spaces.pdf:pdf},
journal = {Proc. Am. Math. Soc.},
number = {2},
pages = {46--50},
title = {{Averaging distances in finite dimensional normed spaces and john's ellipsoid}},
url = {https://www.researchgate.net/publication/241151710_Averaging_distances_in_finite_dimensional_normed_spaces_and_John's_ellipsoid},
urldate={2019-09-11},
volume = {130},
year = {2001}
}
@article{vazquez:l,
author = {Garc{\'{i}}a-V{\'{a}}zquez, Juan C. and Villa, Rafael},
doi = {10.1007/s000130050563},
journal = {Arch. der Math.},
month = {mar},
number = {3},
pages = {222--230},
publisher = {Birkh{\"{a}}user Verlag},
title = {{The average distance property of the spaces $ \ell _\infty ^n ({\Bbb C}) $ and $ \ell _1^n ({\Bbb C}) $}},
url = {http://link.springer.com/10.1007/s000130050563},
urldate={2019-09-11},
volume = {76},
year = {2001}
}
@article{kulshestha:polygons,
abstract = {In this paper we show that the average distance constant of a general polygon which is a subset of an M -space with non-positive curvature can be expressed as the extreme value of either of two nonlinear programs and discuss the practical application of one of these nonlinear programs for the determination of the average distance constant for a polygon in general, and in particular for a planar triangle.1. Kulshestha, D. K., Sag, T. W. & Yang, L. Average distance constants for polygons in spaces with non-positive curvature. Bull. Aust. Math. Soc. 42, 323–333 (1990).},
author = {Kulshestha, Devendra K. and Sag, Tom W. and Yang, Lu},
doi = {10.1017/S0004972700028471},
file = {::},
issn = {0004-9727},
journal = {Bull. Aust. Math. Soc.},
month = {oct},
number = {2},
pages = {323--333},
publisher = {Cambridge University Press},
title = {{Average distance constants for polygons in spaces with non-positive curvature}},
url = {https://www.cambridge.org/core/product/identifier/S0004972700028471/type/journal_article},
urldate={2019-09-11},
volume = {42},
year = {1990}
}
@book{munkres:topology,
author = {Munkres, James R.},
file = {:C\:/Users/Jan/AppData/Local/Mendeley Ltd./Mendeley Desktop/Downloaded/Munkres - 2000 - Topology, Second Edition(2).pdf:pdf},
isbn = {0131816292},
keywords = {munkres:topology},
mendeley-tags = {munkres:topology},
title = {{Topology, Second Edition}},
year = {2000}
}
@article{wolf-energy,
author = {Wolf, Reinhard},
doi = {10.1007/BF02559976},
journal = {Ark. f{\"{o}}r Mat.},
month = {oct},
number = {2},
pages = {387--400},
publisher = {Institut Mittag-Leffler},
title = {{On the average distance property and certain energy integrals}},
url = {http://projecteuclid.org/euclid.afm/1485898558},
urldate={2019-09-11},
volume = {35},
year = {1997}
}
@article{wolf:quasihypermetric-banach,
author = {Wolf, Reinhard},
doi = {10.1007/BF02808179},
journal = {Isr. J. Math.},
month = {nov},
number = {1},
pages = {125--151},
publisher = {Springer-Verlag},
title = {{Averaging distances in real quasihypermetric Banach spaces of finite dimension}},
url = {http://link.springer.com/10.1007/BF02808179},
urldate={2019-09-11},
volume = {110},
year = {1999}
}
@article{nickolas-wolf:quasihypermetric1,
abstract = {Let $(X, d)$ be a compact metric space and let $\mathcal{M}(X)$ denote the space of all finite signed Borel measures on $X$. Define $I \colon \mathcal{M}(X) \to \R$ by \[I(\mu) = \int_X \int_X d(x,y) d\mu(x) d\mu(y),\] and set $M(X) = \sup I(\mu)$, where $\mu$ ranges over the collection of signed measures in $\mathcal{M}(X)$ of total mass 1. The metric space $(X, d)$ is quasihypermetric if for all $n \in \N$, all $\alpha_1, ..., \alpha_n \in \R$ satisfying $\sum_{i=1}^n \alpha_i = 0$ and all $x_1, ..., x_n \in X$, one has $\sum_{i,j=1}^n \alpha_i \alpha_j d(x_i, x_j) \leq 0$. Without the quasihypermetric property $M(X)$ is infinite, while with the property a natural semi-inner product structure becomes available on $\mathcal{M}_0(X)$, the subspace of $\mathcal{M}(X)$ of all measures of total mass 0. This paper explores: operators and functionals which provide natural links between the metric structure of $(X, d)$, the semi-inner product space structure of $\mathcal{M}_0(X)$ and the Banach space $C(X)$ of continuous real-valued functions on $X$; conditions equivalent to the quasihypermetric property; the topological properties of $\mathcal{M}_0(X)$ with the topology induced by the semi-inner product, and especially the relation of this topology to the weak-$*$ topology and the measure-norm topology on $\mathcal{M}_0(X)$; and the functional-analytic properties of $\mathcal{M}_0(X)$ as a semi-inner product space, including the question of its completeness. A later paper [Peter Nickolas and Reinhard Wolf, Distance Geometry in Quasihypermetric Spaces. II] will apply the work of this paper to a detailed analysis of the constant $M(X)$.},
author = {Nickolas, Peter and Wolf, Reinhard},
file = {::},
journal = {Bull. Aust. Math. Soc.},
month = {sep},
title = {{Distance Geometry in Quasihypermetric Spaces. I}},
url = {http://arxiv.org/abs/0809.0740},
urldate={2019-09-11},
volume = {80},
year = {2008}
}
@article{vazquez:max-average,
author = {Garc{\'{i}}a-V{\'{a}}zquez, Juan C. and Villa, Rafael},
doi = {10.1017/S0004972700020748},
file = {:C\:/Users/Jan/Documents/Studium/Rendezvous number/maximum_average_distance_in_complex_finite_dimensional_normed_spaces.pdf:pdf},
issn = {0004-9727},
journal = {Bull. Aust. Math. Soc.},
keywords = {vazquez:max-average},
mendeley-tags = {vazquez:max-average},
month = {aug},
number = {1},
pages = {125--134},
title = {{Maximum average distance in complex finite dimensional normed spaces}},
url = {https://www.cambridge.org/core/product/identifier/S0004972700020748/type/journal_article},
urldate={2019-09-11},
volume = {66},
year = {2002}
}
@article{baronti,
author = {Baronti, M. and Casini, E. and Papini, P.L.},
doi = {10.1016/S0362-546X(99)00112-1},
issn = {0362546X},
journal = {Nonlinear Anal. Theory, Methods Appl.},
month = {oct},
number = {3},
pages = {533--541},
title = {{On average distances and the geometry of Banach spaces}},
url = {https://www.sciencedirect.com/science/article/abs/pii/S0362546X99001121?via%3Dihub},
urldate={2019-09-11},
volume = {42},
year = {2000}
}
@phdthesis{chad,
author = {Chad, Benjamin Michael-John},
file = {:C\:/Users/Jan/Documents/Studium/Rendezvous number/Average distances in Euclidean and other spaces.pdf:pdf},
school = {University of Wollongong},
title = {{Average distances in Euclidean and other spaces}},
type = {Masters thesis},
url = {http://ro.uow.edu.au/theses/2713},
urldate={2019-09-11},
year = {2005}
}
@article{cleary-morris,
author = {Cleary, Joan and Morris, Sidney A},
file = {::},
journal = {Math. Chronicle},
pages = {47--58},
title = {{Numerical geometry ... not numerical topology}},
url={http://www.thebookshelf.auckland.ac.nz/docs/Maths/PDF2/mathschron013-003.pdf},
urldate={2019-09-11},
volume = {13},
year = {1984}
}
@book{cohn,
author = {Cohn, Donald L.},
isbn = {3764330031},
pages = {373},
publisher = {Birkhäuser},
title = {{Measure theory}},
year = {1980}
}
@incollection{gross,
address = {Princeton},
author = {Gross, Oliver},
booktitle = {Adv. Game Theory.},
doi = {10.1515/9781400882014-005},
editor = {Dresher, Melvin and Shapley, Lloyd S. and Tucker, Albert William},
file = {:C\:/Users/Jan/Documents/Studium/Rendezvous number/1964_The Rendezvous Value of a Metric Space.pdf:pdf},
month = {dec},
pages = {49--54},
publisher = {Princeton University Press},
title = {{The Rendezvous Value of a Metric Space}},
url = {http://www.degruyter.com/view/books/9781400882014/9781400882014-005/9781400882014-005.xml},
urldate={2019-09-11},
year = {1964}
}
@article{yost,
author = {Yost, David},
doi = {10.1017/S0004972700005827},
journal = {Bull. Aust. Math. Soc.},
month = {dec},
number = {3},
pages = {331--342},
publisher = {Cambridge University Press},
title = {{Average distances in compact connected spaces}},
url = {https://www.cambridge.org/core/product/identifier/S0004972700005827/type/journal_article},
urldate={2019-09-11},
volume = {26},
year = {1982}
}
@article{pillchshammer:max-gross-stadje,
abstract = {Let X be a compact, connected Hausdorff space and f a real valued, symmetric, continuous function on X × X . Then the Gross-Stadje number r ( X, f ) is the unique real number with the property that for each positive integer n and for all (not necessarily distinct) x 1 ,{\ldots}, x n in X , there exists some x in X such that . This paper solves the following open question in distance geometry: What is the least upper bound g 2 ( R 2 ) of r ( X , d 2 ), where X ranges over all compact, connected subsets of the Euclidean plane with diameter one and where d 2 denotes the squared, Euclidean distance. We show: .},
author = {Pillichshammer, F.},
doi = {10.1017/S0004972700022061},
file = {::},
issn = {0004-9727},
journal = {Bull. Aust. Math. Soc.},
month = {feb},
number = {1},
pages = {109--119},
publisher = {Cambridge University Press},
title = {{A maximal Gross-Stadje number in the Euclidean plane}},
url = {https://www.cambridge.org/core/product/identifier/S0004972700022061/type/journal_article},
urldate={2019-09-11},
volume = {61},
year = {2000}
}
@article{stadje,
author = {Stadje, Wolfgang},
file = {::},
isbn = {81/36030006},
journal = {Arch. Math},
pages = {275--280},
title = {{A property of compact connected spaces}},
url = {https://link.springer.com/content/pdf/10.1007%2FBF01223701.pdf},
urldate={2019-09-11},
volume = {36},
year = {1981}
}
@book{werner:funkana,
abstract = {6., korrigierte Aufl. In: Springer-Online. Jetzt in der sechsten, korrigierten Auflage - Das Lehrbuch für Mathematiker und Physiker, die eine leicht lesbare und gründliche Einführung in die Funktionalanalysis suchen. Aus den Besprechungen: " ... Durch den ökonomischen Aufbau (neun Kapitel, die von normierten Räumen bis zu lokalkonvexen Räumen und Banachalgebren reichen) gelingt es dem Autor, ungewöhnlich viel Material auf engem Raum unterzubringen und die Darstellung dennoch locker zu halten. Hilberträume werden erst relativ spät eingeführt (so da{\ss} die tiefen Einbettungssätze von Sobolev und Rellich vor der Definition der Orthogonalität kommen), aber dadurch wird fast jede Redundanz vermieden. Besonders hervorzuheben sind die Teile über Spektralzerlegung und die Einführung in die Distributionen. Zweihundert Übungsbeispiele sowie Anhänge über Ma{\ss}theorie und Topologie erleichtern das Selbststudium. Ausgezeichnet auch die historischen Bemerkungen und Ausblicke am Schlu{\ss} jedes Kapitels, die auch neueste Ergebnisse berücksichtigen." Monatshefte für Mathematik "Bei dem Springer-Lehrbuch "Funktionalanalysis" von Dirk Werner handelt es sich um einen sehr schönen, gut geschriebenen einführenden Band, der auch Basis von Vorlesungen sein kann, Dozenten und Studierenden als Nachschlagewerk dienen mag und von dem sich Teile als Grundlage für Funktionalanalysis-Seminare verwenden lassen. Das Buch deckt den gesamten "klassischen" Stoff einer zweisemestrigen Funktionalanalysis-Vorlesung für Diplom-Mathematiker und -Physiker sowie Lehramtskandidaten ab, bietet aber mehr: Es setzt Akzente, wie man sie bei anderen Einführungen in die Funktionalanalysis bisher nicht oder nicht in diesem Umfang gefunden hat. Generell liegen die Stärken des Buches in den vielen durchgerechneten Beispielen und Anwendungen; dazu gibt es in jedem der neun Kapitel einen Abschnitt mit interessanten Übungen. Die jedes Kapitel abschlie{\ss}enden "Bemerkungen und Ausblicke" sind besonders willkommen: Hier geht der Aut.},
author = {Werner, Dirk.},
isbn = {9783540725367},
keywords = {werner:funkana},
mendeley-tags = {werner:funkana},
publisher = {Springer-Verlag Berlin Heidelberg},
title = {{Funktionalanalysis}},
year = {2007}
}
@article{hinrichs:banach1,
abstract = {A Banach space X has the average distance property (ADP) if there exists a unique real number r such that for each positive integer n and all x\,... ,x n in the unit sphere of X there is some x in the unit sphere of X such that i {\pounds} \\x k-x\\=r. It is known that h and /<» have the ADP, whereas l p fails to have the ADP if 1 ^ p < 2. We show that l p also fails to have the ADP for 3 < p ^ oo. Our method seems to be able to decide also the case 2 < p < 3, but the computational difficulties increase as p comes closer to 2.},
author = {Hinrichs, Aicke},
doi = {10.1017/S0004972700018530},
file = {::},
title = {{The average distance property of classical banach spaces}},
url = {https://www.cambridge.org/core/journals/bulletin-of-the-australian-mathematical-society/article/average-distance-property-of-classical-banach-spaces/252261D30D15D385B27A118BB1E7CD52},
urldate={2019-09-11},
year = {2019}
}
@article{helly,
author = {Helly, Edmond},
journal = {Jahresbericht der Deutschen Mathematiker-Vereinigung},
pages = {175--176},
title = {{{\"{U}}ber Mengen konvexer K{\"{o}}rper mit gemeinschaftlichen Punkte.}},
url = {https://eudml.org/doc/145659},
urldate={2019-09-11},
volume = {32},
year = {1923}
}
@article{lin,
abstract = {Let A; d† be a bounded metric space. A positive real number a is said to be a rendezvous number of A if for any n 2},
journal={Arch. Math.},
year={1997},
volume={68},
author = {Lin, Pei-Kee},
file = {::},
title = {{The average distance property of Banach spaces}},
url = {https://link.springer.com/content/pdf/10.1007%2Fs000130050082.pdf},
urldate={2019-09-11}
}
@article{wolf:finite-real-banach,
author = {Wolf, Reinhard},
doi = {10.1017/S0004972700013927},
journal = {Bull. Aust. Math. Soc.},
month = {feb},
number = {1},
pages = {87--101},
publisher = {Cambridge University Press},
title = {{On the average distance property in finite dimensional real Banach spaces}},
url = {https://www.cambridge.org/core/product/identifier/S0004972700013927/type/journal_article},
urldate={2019-09-11},
volume = {51},
year = {1995}
}
@article{nickolas-wolf:quasihypermetric2,
author = {Nickolas, Peter and Wolf, Reinhard},
doi = {10.1002/mana.200710206},
file = {::},
issn = {0025584X},
journal = {Math. Nachrichten},
keywords = {31C45,Compact metric space,MSC (2010) Primary 51K05,Secondary: 54E45,distance geometry,finite metric space,geometric constant,metric embedding,quasihypermetric space,signed measure,signed measure of mass zero,space of negative type,spaces of measures},
month = {feb},
number = {2-3},
pages = {332--341},
publisher = {John Wiley and Sons, Ltd},
title = {{Distance geometry in Quasihypermetric spaces. II}},
url = {https://onlinelibrary.wiley.com/doi/abs/10.1002/mana.200710206},
urldate={2019-09-11},
volume = {284},
year = {2011}
}
@book{rudin:functional-analysis,
abstract = {Second edition. This classic text is written for graduate courses in functional analysis. This text is used in modern investigations in analysis and applied mathematics. This new edition includes up-to-date presentations of topics as well as more examples and exercises. New topics include Kakutani's fixed point theorem, Lamonosov's invariant subspace theorem, and an ergodic theorem. This text is part of the Walter Rudin Student Series in Advanced Mathematics. pt. I. General theory. 1. Topological vector spaces -- 2. Completeness -- 3. Convexity -- 4. Duality in Banach spaces -- 5. Some applications -- pt. II. Distributions and Fourier transforms. 6. Test functions and distributions -- 7. Fourier transforms -- 8. Applications to differential equations -- 9. Tauberian theory -- pt. III. Banach algebras and spectral theory. 10. Banach algebras -- 11. Commutative Banach algebras -- 12. Bounded operators on a Hilbert space -- 13. Unbounded operators.},
author = {Rudin, Walter},
isbn = {0070619883},
keywords = {rudin:functional-analysis},
mendeley-tags = {rudin:functional-analysis},
pages = {424},
title = {{Functional analysis}}
}
@article{humble:rendezvous,
abstract = {//static.cambridge.org/content/id/urn%3Acambridge.org%3Aid%3Aarticle%3AS0025557200183196/resource/name/firstPage-S0025557200183196a.jpg},
author = {Humble, Steve},
doi = {10.1017/S0025557200183196},
file = {::},
issn = {0025-5572},
journal = {Math. Gaz.},
month = {jul},
number = {524},
pages = {287--290},
publisher = {Cambridge University Press},
title = {{Rendezvous constants}},
url = {https://www.cambridge.org/core/product/identifier/S0025557200183196/type/journal_article},
volume = {92},
year = {2008}
}
@article{farkas:potential-theory,
abstract = {We analyze relations between various forms of energies (reciprocal capacities), the transfinite diameter, various Chebyshev constants and the so-called rendezvous or average number. The latter is originally defined for compact connected metric spaces (X,d) as the (in this case unique) nonnegative real number r with the property that for arbitrary finite point systems {x1,...,xn} in X, there exists some point x in X with the average of the distances d(x,xj) being exactly r. Existence of such a miraculous number has fascinated many people; its normalized version was even named "the magic number" of the metric space. Exploring related notions of general potential theory, as set up, e.g., in the fundamental works of Fuglede and Ohtsuka, we present an alternative, potential theoretic approach to rendezvous numbers.},
author = {Farkas, Balint and Revesz, Szilard Gy.},
file = {::},
journal = {S. Mh Math},
month = {mar},
number = {309},
title = {{Potential theoretic approach to rendezvous numbers}},
url = {https://link.springer.com/article/10.1007%2Fs00605-006-0397-5},
urldate={2019-09-11},
volume = {148},
year = {2006}
}
@book{ferguson:decistion-theory,
author = {Ferguson, Thomas S.},
isbn = {9781483182537},
pages = {396},
publisher = {Academic Press},
title = {{Mathematical statistics a decision theoretic approach}},
year = {1967}
}
